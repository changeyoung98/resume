% !TEX program = xelatex

\documentclass{resume}
\usepackage{zh_CN-Adobefonts_external} % Simplified Chinese Support using external fonts (./fonts/zh_CN-Adobe/)
%\usepackage{zh_CN-Adobefonts_internal} % Simplified Chinese Support using system fonts

\begin{document}
\pagenumbering{gobble} % suppress displaying page number

\name{陈志扬}

\basicInfo{
  \email{1792266893@qq.com} \textperiodcentered\ 
  \phone{(+86) 155-8877-0792} \textperiodcentered}
  
\section{\faGraduationCap\  教育背景}
\datedsubsection{\textbf{上海交通大学}, 上海}{2016 -- 至今}
\textit{本科生在读}\ 软件工程, 预计 2020 年 7 月毕业

\section{\faUsers\ 项目/科研/交流经历}
\subsection{\textbf{项目经历}}
%\begin{个人项目}
\begin{itemize}
  \item \textbf{“杀人”飞行器的行动模拟}:计算机图形学大作业,使用C++/OpenGL完成的动画效果,模拟杀人飞行器的行动及决策、搜寻目标、与“反杀人”飞行器的对抗。
  \item \textbf{文件系统}:计算机系统工程大作业,四次迭代完成的yfs文件系统。
  \item \textbf{网上书店}:web开发技术大作业,四次迭代完成的可以进行增删改查的网上书店,分为普通用户和管理员两种身份登录。
  \item \textbf{基于FPGA的CPU}:数字部件设计课程作业,完成了一个流水线CPU。
  \item \textbf{简易数据库}:用C++编写的简易数据库,可以进行增删改查。
\end{itemize}
%\end{onehalfspacing}

%\begin{团队项目}
\begin{itemize}
  \item \textbf{基于VR环境的简笔画模型检索系统}:人机交互与设计课程大作业,三人团队项目。使用VRTK在Unity开发平台进行开发,以VR环境下中学实验为背景,开发出的在VR环境中以简笔画形式进行实验器材检索获取的系统。该课程作业获得98分,在此课程中排名第一。
  \item \textbf{基于VR环境的折纸模拟系统}:游戏开发与设计课程大作业,三人团队项目。参考论文《》和项目,在Unity开发平台进行纸张的物理仿真引擎的搭建,并将其移植入VR环境,为使用者提供VR交互体验。该课程作业获得96分,在此课程中排名第二。
  \item \textbf{“慧眼识踪”}:暑期软件开发团队项目,完成了一个web应用。该应用可以选择并连接实时摄像头,对画面进行截图、框选目标人物,并对该人物在其余的监控视频中进行检索,找到目标人物。
  \item \textbf{基于鱼眼摄像头的三维重建}:计算机视觉团队大作业,根据给定的论文及源代码对重建过程进行优化,得到更好的建模结果。
  \end{itemize}
%\end{onehalfspacing}

\subsection{\textbf{科研经历}}
\role{上海交通大学数字艺术实验室}{指导老师 杨旭波教授}
目前参与的项目:Magic Toon科研小组
%\begin{参与Magic Toon科研小组}
\begin{itemize}
  \item 开发了一个用来分割数据集的工具
\end{itemize}
%\end{onehalfspacing}
\datedsubsection{\textbf{实习经历}}{2019年7月 -- 至今}
\role{依图科技}{硬件研发部}
后台开发全职实习生,在硬件部门的AINVR组进行项目的CI环境搭建与维护、web端后台开发的工作。
\datedsubsection{\textbf{暑期交流经历}}{2018年8月}
\role{牛津大学}{2周}
此项目为上海交通大学“菁莪本科生暑期海外交流”项目。在全校大二、大三的学生共选出15人,全额资助(总资助金额达40,000元)前往牛津大学进行为期两周的学习交流。


% Reference Test
%\datedsubsection{\textbf{Paper Title\cite{zaharia2012resilient}}}{May. 2015}
%An xxx optimized for xxx\cite{verma2015large}
%\begin{itemize}
%  \item main contribution
%\end{itemize}

\section{\faCogs\ IT 技能}
% increase linespacing [parsep=0.5ex]
\begin{itemize}[parsep=0.5ex]
  \item 编程语言: C/C++, Python, Java, JavaScript等
  \item 平台: Windows/Linux
\end{itemize}

\section{\faHeartO\ 获奖情况}
\datedline{国家励志奖学金}{2017年}
\datedline{上海交通大学C等奖学金}{2017、2018年}
\datedline{美国大学生数学建模竞赛H奖}{2018年}
\datedline{上海交通大学“三好学生”}{2017、2018年}
\datedline{上海交通大学“五四评优”优秀团员}{2017、2018年}

\section{\faInfo\ 其他}
% increase linespacing [parsep=0.5ex]
\begin{itemize}[parsep=0.5ex]
  \item GitHub: https://github.com/changeyoung98
  \item 语言: 英语 较为熟练(六级 629)
\end{itemize}

%% Reference
%\newpage
%\bibliographystyle{IEEETran}
%\bibliography{mycite}
\end{document}
